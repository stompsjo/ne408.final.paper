\section{Introduction}

Since Sputnik was launched into space in 1957, humans have strived to be a spacefaring species.  Successful missions landed men on the Moon and sent satellites to probe the outer reaches of the Solar System. Largely due to tenuous funding and an uncertain purpose, space travel in the 21st century has been a quiet endeavor. The United States National Aeronautics and Space Administration (NASA) has largely been focused on robotic studies of other planets, developing advanced space technologies, and maintaining current infrastructure like the International Space Station (ISS) and the Hubble Space Telescope.

The astronautics industry life cycle tends to be punctuated by events like the establishment of a new military branch in the United States: The Space Force \cite{space-force}. Civilian and scientific interest in space travel has been dominated by the desire to further scientific understanding of the universe and the commercial desire to extract extraterrestrial resources for use on Earth. Because of historical challenges in publicly funding space exploration, there has been an advent of private space companies funded by investors and enterprises like SpaceX, Virgin Galactic, and Blue Origin. The economic advantage of these companies is their ability to self-fund space technology and conduct contract work \cite{space-contract} with less reliance on public funding.

In recent years there has been a renewed interest in space exploration propelled by private interests, that are returning public interests to the focus of manned missions, particularly deep space missions to other planets instead of robots or satellite probes \cite{nasa-landers}. This effort seems most notable in SpaceX, who have very publicly expressed their interest in leading missions that culminate in permanent settlements on extraterrestrial bodies \cite{spacex-mars-press}. This unprecedented feat, if successful, would lead to a new era in human history that previously only lived in the realm of science fiction.

Even if public interest remains strong, this process will require significant strides ahead of current space technology. Manned missions require additional payloads for life-support that would normally not be required for deep space missions. This extra payload means new rockets and propulsion systems are needed for sending astronauts on long journeys. NASA has detailed its expectations for novel and exotic space propulsion systems, including when they are expected to be viable \cite{nasa-propulsion}. Among the potential systems are spacecraft driven by nuclear reactors. Regardless of what innovative technology ends up being used, any manned spacecraft will require shielding to protect humans from the biological effects of space travel, particularly harmful radiation. This paper presents a literature review on the existing research and requirements for designing spacecraft suitable for human deep space exploration. Reviews of the biological effects of space radiation and the existing shielding strategy for spaceship design are included.
